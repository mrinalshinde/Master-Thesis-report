

\newpage
\thispagestyle{empty}
%\begin{framed}
\begin{center}
\Huge\emph{Kurzfassung}
\end{center}
\medskip
\noindent

Der kommende Mobilfunkstandart 5G, welcher von der 3GPP erarbeitet wird, verwendet unter anderem Frequenzen im Millimeter Wellen Bereich. Bei diesen kurzen Wellenlängen sind herkömmliche Testmethoden, welche Kabel Konnektoren am zu testenden Gerät voraussetzten auf Grund des hohe Integrationsgrad moderner aktiver Antennen Systeme nicht praktikabel. Daher werden in Zukunft Störstrahlungsmessungen und Konformitätstests kabellos durchgeführt. Es wird die gesamte abgestrahle Leistung als Metrik verwendet. Im Rahmen dieser Arbeit sollen alle Aspekte der nicht verkabelten Störstrahlungsmessung untersucht werden. Anschließend soll ein konkreter Vorschlag für die ideale Durchführung dieser Messung erarbeitet werden. Das beinhaltet eine allgemeine Übersicht der Schwierigkeiten und Lösungen einer jeden nicht verkabelten Messung. Zudem muss ein ideales sphärisches Abtastnetz gefunden werden. Hierfür soll eine statistische Simulation eingesetzt werden. Außerdem soll mit dem Einsatz von verschiedenen Simulationswerkzeugen und einer Messkampange der minimale Abstand für eine korrekte Störstrahlungsmessung gefunden werden.
