

\newpage
\thispagestyle{empty}
%\begin{framed}
\begin{center}
\Huge\emph{Kurzfassung}
\end{center}
\medskip
\noindent

Der kommende Mobilfunkstandard 5G NR, der von 3GPP entwickelt wird, nutzt Frequenzen im Millimeterwellenbereich.
Bei diesen kurzen Wellenlängen sind herkömmliche Testmethoden, die Kabelkonnektoren in der Benutzerausrüstung benötigen, aufgrund des hohen Integrationsgrades moderner aktiver Antennensysteme impraktikabel.
Daher müssen in Zukunft Konformitätstests und Störstrahlungsmessungen für den Millimeterwellenbereich über die Luft, unter Verwendung der Metrik TRP, durchgeführt werden.
Die Grenzwerte für elektromagnetische Emissionen sind im Vergleich zwischen der Europäischen Union und den Vereinigten Staaten von Amerika dargelegt.
Erst werden theoretische Hintergründe in den erforderlichen Bereichen der Antennentheorie, räumlicher Abtastung, sphärischer Abtastnetze, sphärischer Quadraturen, Stochastik und Regression erklärt.\\

Anschließend folgt die Untersuchung aller Aspekte von TRP-Messungen über die Luft.
Angefangen bei der Messkammerauswahl, um eine ausreichende Dynamik zu ermöglichen und Quiet Zone zu erreichen, der Kalibrierung eines Messplatzes, der Bewertung des Dynamikbereichs anhand des Link Budget und der Auswahl der idealen Detektoreinstellung für das verwendete Messgerät.
Anschließend folgt die Erarbeitung eines konkreten Vorschlags zur idealen Durchführung der TRP-Messung.
Einerseits wird der minimal erforderliche Messabstand und die ideale Messantenne um eine einfache Messung durch kurze Messabstände, der über die Luft TRP-Messung, untersucht.
Hierzu folgt die Entwickelung und Durchführung einer Simulation der getesteten Antenne, der Messantenne und des Messradius.
Mit verschiedenen Parametersätzen wird die Tendenz ermittelt, dass Messantennen mit geringerer Direktivität bei gleichem Messabstand weniger Fehler verursachen. Dies wird durch eine Messkampagne bestätigt.\\

Andererseits werden der sphärische Abtasttyp und die Messdichte mithilfe einer statistischen Simulation untersucht, welche rechteckige Antennenarrays mit zufälligen Steering-Vektoren verwendet.
Es ist wichtig, einfach realisierbare Abtastungen mit so wenig Punkten auf der Kugel wie möglich zu kombinieren, um in kürzester Zeit eine bestimmte Genauigkeit der TRP-Messung zu erzielen.
Hier hat sich der, in dieser Arbeit entwickelte, Referenzwinkelkorrekturfaktor als die optimale Methode erwiesen, um einen öffnungsunabhängigen Messfehler mit einer gleichförmigen sphärischen Abtastung mit ausgedünnter Abtastung an den Polen, zu erreichen.
Dies wurde durch eine Regression bewiesen.
