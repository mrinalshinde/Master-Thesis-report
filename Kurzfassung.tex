

\newpage
\thispagestyle{empty}
%\begin{framed}
\begin{center}
\Huge\emph{Kurzfassung}
\end{center}
\medskip
\noindent

Der kommende Mobilfunkstandard 5G NR, der von 3GPP entwickelt wird, nutzt Frequenzen im Millimeterwellenbereich.
Bei diesen kurzen Wellenlängen sind herkömmliche Testmethoden, die Kabelverbinder in der Benutzerausrüstung benötigen, aufgrund des hohen Integrationsgrades moderner aktiver Antennensysteme, unpraktikabel.
Daher werden in Zukunft Konformitätstests und Störstrahlungsmessungen für den Millimeterwellenbe-reich über die Luft, unter Verwendung der Metrik TRP, durchgeführt.
Die Grenzwerte für elektromagnetische Emissionen werden im Vergleich zwischen der Europäischen Union und den Vereinigten Staaten von Amerika dargelegt.
Erst werden theoretische Hintergründe in den erforderlichen Bereichen der Antennentheorie, räumlicher Abtastung, sphärischer Abtastnetze, sphärischer Quadraturen, Stochastik und Regression erklärt.\\

Dann werden im Rahmen dieser Arbeit alle Aspekte von TRP-Messungen über die Luft untersucht.
Angefangen bei der Messkammerauswahl, um eine ausreichende Dynamik und Quiet Zone zu erreichen, der Kalibrierung eines Messplatzes, der Bewertung des Dynamikbereichs anhand des Link Budget und der Auswahl der idealen Detektoreinstellung für das verwendete Messgerät.
Anschließend wird ein konkreter Vorschlag zur idealen Durchführung der TRP-Messung erarbeitet.
Einerseits wird der minimal erforderliche Messabstand und die ideale Messantenne um eine einfache Messung durch kurze Messabstände, der über die Luft TRP-Messung, untersucht.
Hierzu wird eine Simulation der getesteten Antenne, der Messantenne und des Messradius entwickelt und durchgeführt.
Mit verschiedenen Parametersätzen wird die Tendenz ermittelt, dass Messantennen mit geringerer Direktivität bei gleichem Messabstand weniger Fehler verursachen, dies wird durch eine Messkampagne bestätigt.\\

Andererseits werden der sphärische Abtasttyp und die Messdichte mithilfe einer statistischen Simulation untersucht, welche rechteckige Arrays mit zufälligen Steering-Vektoren verwendet.
Es ist wichtig, einfach realisierbare örtliche Abtastungen mit so wenig Punkten auf der Kugel wie möglich zu kombinieren, um in kürzester Zeit eine bestimmte Genauigkeit zu erzielen.
Hier hat sich der entwickelte Referenzwinkelkorrekturfaktor als die optimale Methode erwiesen, um einen öffnungsunabhängigen Messfehler mit einer gleichförmigen sphärischen Abtastung mit ausgedünnter Abtastung an den Polen, zu erreichen.
Dies wurde durch eine Regression bewiesen.
