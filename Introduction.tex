\chapter{Introduction}

In this chapter the motivation and aim of this work are phrased. Furthermore, the current emission limits and the necessary terms are presented.

\section{Motivation}

\ac{3GPP} \ac{5G} is occupying \acp{mmW} in \ac{FR2} with frequencies $\ge\SI{28}{\giga\hertz}$ \cite{trp}. With this development big and highly integrated  \acp{AAS} will be needed. That makes the conducted conformance test approach with antenna connectors in the \ac{UE} impossible. The equivalent metric to conducted measurements for \ac{RSE} is now \ac{TRP} and the measurement is carried out \ac{OTA}.

%Higher frequencies imply shorter wavelengths. Multi-antenna arrays at mmWave frequencies will help overcome signal propagation issues and deliver directional antennas with higher gain. With shorter wavelengths, antenna elements can be spaced more tightly, resulting in extremely compact arrays. Many vendors even opt to develop arrays that integrate into ICs. Of course, highly integrated ICs have no place to probe and no place to put connectors. A consequence of this integration is it has become impractical to use traditional RF connectors between the radio circuit and the antenna, bringing the need for OTA tests. https://www.testandmeasurementtips.com/why-5g-is-going-to-over-the-air-testing-faq/

\section{Aim}

The aim of this work is to find an optimal solution for conformance  \ac{TRP}-measurements using \acp{AC}. In dependence of the dimension of the \ac{DUT} an ideal grid and measurement distance for a wanted precision shall be found. To achieve that, theoretical knowledge shall be projected on simulations and the outcome shall be proofed in real measurements.

\section{Legal Emission Limits}
\label{sec:legem}

In this section the official \ac{EM} emission limits are acknowledged by the example of the regulation in the \ac{USA} and the \ac{EU}. The \ac{ITU} provides a recommendation \cite{seitur} about the emission limits for the national regulators. These regulators, in the \ac{EU} the \ac{CEPT} and in the \ac{USA} the \ac{FCC}, publish with the help of dedicated standardization organizations, e.g. \ac{ETSI} (\ac{EU}), \ac{ANSI} (\ac{USA}) or other \acp{SDO}, limits which become law; \cite{ceptercrec}, \cite{ansi}. The \ac{3GPP}, the \ac{IEEE} or other \acp{SDO} develop a new telecommunication standard and adapt to these laws.\\
To quantify emission limits the following terms are declared regarding \cite{seitur} and \cite{ctiaat}:

\begin{itemize}
\item An \textbf{\acf{UnE}} consists of \ac{SE} and \ac{OOBE}.
\item A \textbf{\acf{SE}} is an emission on a frequency or frequencies, which are outside the necessary bandwidth and the level of which may be reduced without affecting the corresponding transmission of information. \acl{SE} include harmonic
emissions, parasitic emissions, intermodulation products and frequency conversion products but exclude \ac{OOBE}.
\item An \textbf{\acf{OOBE}} is an emission on a frequency or frequencies immediately outside the necessary bandwidth which results from the modulation process, excluding \aclp{SE}.
\item The \textbf{\acf{OBW}} is the bandwidth wherein $\SI{99}{\percent}$ of the power is radiated.
\item The \textbf{\acf{EBW}} is used by the \ac{FCC} and describes a bandwidth which is surrendered out of that tow points in the spectra lying $\SI{26}{\decibel}$ under the power of the \ac{OBW}. 
\item \textbf{\acf{TRP}} is the \ac{OTA} equivalent for measuring connected and describes all power radiated by a device. It is the average spherical \ac{EIRP}.
\end{itemize}


In table \ref{tab:legusa} the emission limits from \cite{ceptercrec}, \cite{ansi} and \cite{fcc} are summarized. The used terms are illustrated in fig. \ref{fig:sem}.

\begin{table}
\centering
\caption{Legal Emission Limits}
\label{tab:legusa}
\begin{tabular}{|c|c|c|c|}
\hline
$f_c$  &  $f_l$ & $f_h$ & $p_e$ \\\hline
\multicolumn{4}{|c|}{\textbf{USA}} \\\hline
$\SI{9}{\kilo\hertz}-\SI{10}{\giga\hertz}$ & $\SI{9}{\kilo\hertz}$ & $10f_c\ \text{or}\ \SI{40}{\giga\hertz}$ & $\SI{-13}{\decibelm\per\mega\hertz}$ \\\hline
$\SI{10}{\giga\hertz}-\SI{30}{\giga\hertz}$ & $\SI{9}{\kilo\hertz}$ & $5f_c\ \text{or}\ \SI{100}{\giga\hertz}$ & $\SI{-13}{\decibelm\per\mega\hertz}\ \text{or}\ 1\% \text{EBW}$ \\\hline
$\SI{30}{\giga\hertz}-\dots$ & $\SI{9}{\kilo\hertz}$ & $5f_c\ \text{or}\ \SI{200}{\giga\hertz}$ & $\SI{-13}{\decibelm\per\mega\hertz}\ \text{or}\ 1\% \text{EBW}$ \\\hline
\multicolumn{4}{|c|}{\textbf{EU}} \\\hline
$\SI{9}{\kilo\hertz}-\SI{100}{\mega\hertz}$ & $\SI{9}{\kilo\hertz}$ & $\SI{1}{\giga\hertz}$ & $\SI{-36}{\decibelm\per \SI{100}{\kilo\hertz}}$ \\\hline
$\SI{100}{\mega\hertz}-\SI{300}{\mega\hertz}$ & $\SI{9}{\kilo\hertz}$ & $10f_c$ & $\SI{-36}{\decibelm\per \SI{100}{\kilo\hertz}}$ \\\hline
$\SI{300}{\mega\hertz}-\SI{600}{\mega\hertz}$ & $\SI{30}{\mega\hertz}$ & $\SI{3}{\giga\hertz}$ & $\SI{-36}{\decibelm\per \SI{100}{\kilo\hertz}}$ \\\hline
$\SI{600}{\mega\hertz}-\SI{1}{\giga\hertz}$ & $\SI{30}{\mega\hertz}$ & $5f_c$ & $\SI{-36}{\decibelm\per \SI{100}{\kilo\hertz}}$ \\\hline
$\SI{1}{\giga\hertz}-\SI{5.2}{\giga\hertz}$ & $\SI{30}{\mega\hertz}$ & $5f_c$ & $\SI{-30}{\decibelm\per\mega\hertz}$ \\\hline
$\SI{5.2}{\giga\hertz}-\SI{7.25}{\giga\hertz}$ & $\SI{30}{\mega\hertz}$ & $\SI{26}{\giga\hertz}$ & $\SI{-30}{\decibelm\per\mega\hertz}$ \\\hline
$\SI{7.25}{\giga\hertz}-\dots$ & $\SI{30}{\mega\hertz}$ & $2f_c$ & $\SI{-13}{\decibelm\per\mega\hertz}\ \text{and}\ \SI{-10}{\decibelm\per \SI{100}{\mega\hertz}}$ \\\hline
\end{tabular}
\end{table}

\begin{figure}
\centering
\def\svgwidth{0.9\textwidth}
\input{Bilder/EmissionMask.pdf_tex}
\caption{Spurious Emissions Mask}
\label{fig:sem}
\end{figure}








