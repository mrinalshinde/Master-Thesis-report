%%%%%%%%%%%%%%%%%%%%%%%%%%%%%%%%%%%%%%%%%%%%%%%%%%%%%%%%%%%%%%%%%%%%%%%%%%%%%%%
%% Descr:       Vorlage für Berichte der DHBW-Karlsruhe, Erklärung
%% Author:      Prof. Dr. Jürgen Vollmer, vollmer@dhbw-karlsruhe.de
%% $Id: erklaerung.tex,v 1.6 2016/03/16 12:51:09 vollmer Exp $
%% -*- coding: utf-8 -*-
%%%%%%%%%%%%%%%%%%%%%%%%%%%%%%%%%%%%%%%%%%%%%%%%%%%%%%%%%%%%%%%%%%%%%%%%%%%%%%%

% In Bachelorarbeiten muss eine schriftliche Erklärung abgegeben werden.
% Hierin bestätigen die Studierenden, dass die Bachelorarbeit, etc.
% selbständig verfasst und sämtliche Quellen und Hilfsmittel angegeben sind. Diese Erklärung
% bildet das zweite Blatt der Arbeit. Der Text dieser Erklärung muss auf einer separaten Seite
% wie unten angegeben lauten.

\newpage
\thispagestyle{empty}
%\begin{framed}
\begin{center}
\Huge\emph{Eidesstattliche Erklärung}
\end{center}
\medskip
\noindent
Ich versichere hiermit, dass ich meine \Titel\ mit
dem Thema: \glqq \Was \grqq{ }selbständig verfasst und keine anderen als die angegebenen Quellen und Hilfsmittel benutzt habe. Ich versichere zudem, dass die eingereichte elektronische Fassung mit der gedruckten Fassung übereinstimmt.\todo{muss man eine elektronische Fassung abgeben}

\vspace{3cm}
\noindent
\underline{\hspace{4cm}}\hfill\underline{\hspace{6cm}}\\
Ort~~~~~Datum\hfill Unterschrift\hspace{4cm}
%\end{framed}

\vfill
%\emph{Sofern von der Ausbildungsstätte ein Sperrvermerk gewünscht %wird, ist folgende Formulierung
%zu verwenden:}
%\begin{framed}
%\begin{center}
%\Large\bfseries Sperrvermerk
%\end{center}
%\medskip
%\noindent
%Der Inhalt dieser Arbeit darf weder als Ganzes noch in Auszügen %Personen
%auerhalb des Prüfungsprozesses und des Evaluationsverfahrens %zugänglich gemacht
%werden, sofern keine anders lautende Genehmigung der %Ausbildungsstätte vorliegt.
%\end{framed}

%%%%%%%%%%%%%%%%%%%%%%%%%%%%%%%%%%%%%%%%%%%%%%%%%%%%%%%%%%%%%%%%%%%%%%%%%%%%%%%
\endinput
%%%%%%%%%%%%%%%%%%%%%%%%%%%%%%%%%%%%%%%%%%%%%%%%%%%%%%%%%%%%%%%%%%%%%%%%%%%%%%%
