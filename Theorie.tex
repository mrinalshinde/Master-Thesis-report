\chapter{Theoretical Background}

This chapter will give a brief overview about the underlying theoretical background in the fields of antenna theory, spatial sampling and will some methods to derive the \ac{TRP} out of spherical measurements.

\section{Antenna Field Regions}
\begin{figure}[H]
\centering
\def\svgwidth{0.6\textwidth}
\input{Bilder/AntennaApature.pdf_tex}
\caption{An arbitrary radiating aperture in the $xy$ plane. \cite{7942128}}
\label{arbaperturexy}
\end{figure}

\ac{FF}-distance and \ac{NF}-distance are derivable from the phase fluctuation due to the maximum diameter of the antennas aperture \cite{7942128} in a distance $R$ from the antennas phase center. The phase fluctuation is given by the different length of $r$ and $R$, as you can see in figure \ref{arbaperturexy}. The maximum runtime difference is found at the minimum radius of the circle enclosing the aperture at $\sfrac{D}{2}$. The field boundaries are derived from the Taylor series of the function of the phase in dot $P$. \cite{7942128}


\subsection{Near-Field}

The reactive \acf{NF} is according to \cite{balanis} \glqq that portion of the near-field region immediately surrounding the antenna wherein the reactive field predominates.\grqq{ }The Boundary of that field region is derived for big apertures by a line source and the underlying phase fluctuation of  $\Delta\phi = \sfrac{\pi}{8} \ \widehat{=}\  22.5^\circ$. The boundary is:
\begin{equation}
r_{\text{NF}} = 0.62\sqrt{\frac{D^3}{\lambda}}
\end{equation}
After that the radiating \ac{NF} (Fresnel) -region follows, which is defined as \glqq that region of the field of an antenna between the reactive near-field region and the far-field region wherein radiation fields predominate and wherein the angular field distribution is dependent upon the distance from the antenna.\grqq

\subsection{Far-Field}

The fraunhofer distance is derived from optical considerations. The commonly used terminology is \acf{FF}-Distance. According to \cite{balanis} the \ac{FF}-distance is \glqq that region of the field of an antenna where the angular field distribution is essentially independent of the distance from the antenna.\grqq \\
The amount of independency is derived from the phase fluctuation in the distance $r_{\text{FF}}$ \cite{19510}:

\begin{equation}
\Delta\phi = \frac{\pi D^2}{4\lambda\cdot r_{\text{FF}}} ,\ D\gg\lambda 
\end{equation}

The commonly used maximum phase fluctuation is $\Delta\phi = \sfrac{\pi}{8} \ \widehat{=}\  22.5^\circ$. With that the well know formula
\begin{equation}
r_{\text{FF}} = \frac{2D^2}{\lambda}
\end{equation}

can be derived.

\subsection{Derat Distance}

The Derat distance is that distance, in which the main beam of an antenna is in \ac{FF} condition. For the understanding of the Derat distance $r_{\text{Dr}}$ other considerations need to take place: It is about spherical modes described by spherical Hankel functions  of the second kind. \cite{8393926} \cite{hansen}\\
In \cite{8393926} it is shown, that higher order modes than 

\begin{equation}
N = \lceil 1.0252\cdot\left(\frac{\pi D}{\lambda}\right)^{0.8633} \rceil
\end{equation}

have a maximum influence to the \ac{EIRP} of $\SI{0.5}{\decibel}$. With that the Derat distance can be derived to:

\begin{equation}
r_{\text{Dr}} = \lambda\left(\frac{\pi D}{\lambda}\right)^{0.8633}\left(0.1673\left(\frac{\pi D}{\lambda}\right)^{0.8633}+0.1632\right)
\end{equation}

\subsection{Example: Standard Gain Horn}

$\SI{20}{\decibel}$ \ac{SGH} at $\SI{28}{\giga\hertz}$

\section{Spatial Sampling Approaches}

\subsection{Constant Step}
Constant Step, Constant Step with swept Azimuth,

\subsection{Constant Density}

Constant Density

\subsection{Other}

Spiral Scan
Cardinal Cuts with or without Pattern Multiplication

\section{Total Radiated Power Integrals in Comparison}

\subsection{Sinus Theta}

\subsection{Jacobi}

\subsection{Clenshaw-Curtis}

\section{Measurement Distance}