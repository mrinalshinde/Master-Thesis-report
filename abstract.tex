

\newpage
\thispagestyle{empty}
%\begin{framed}
\begin{center}
\Huge\emph{Abstract}
\end{center}
\medskip
\noindent

The upcoming mobile communication standard 5G NR, which is developed by the 3GPP, uses frequencies in the millimetre range. At these short wavelengths, conventional test methods which require cable connectors on the device under test are impractical due to the high degree of integration of modern active antenna systems. Therefore, in the future, radiated spurious emissions measurements and conformity tests will be carried out over the air using the metric total radiated power. Within the scope of this thesis all aspects of over the air radiation measurements shall be investigated. Subsequently, a concrete proposal for the ideal implementation of this total radiated power measurement shall be developed. This includes a general overview of the difficulties and solutions of over the air measurements in general. In addition, an ideal spherical scanning network must be found. For this a statistical simulation shall be used. In addition, with the use of different simulation tools and a measuring campaign, the minimum range length for a correct total radiated power measurements shall be found.
