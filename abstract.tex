

\newpage
\thispagestyle{empty}
%\begin{framed}
\begin{center}
\Huge\emph{Abstract}
\end{center}
\medskip
\noindent As more and more manufacturers incorporate wireless communication into devices, the interference to purposeful communication grows, which gives rise to a risk to the proper operation of critical devices such as wireless medical devices. Wireless coexistence is concerned with a wireless device's ability to perform a task in a given shared space where other wireless systems are performing their tasks and may or may not be using the same set of regulations. Wireless coexistence testing, therefore, helps to mitigate this risk of interference and provide safe products for users worldwide. This thesis discusses the novel approach for coexistence testing for devices with and without cable connectors. The technologies that the system should be evaluated against include Wi-Fi\texttrademark{}, Bluetooth\textregistered{}, and others. This work is proposed only for In-Band testing where wireless communications are degraded as well as the threshold where communications stop. The new approach is the normalized measurement which is a non-conducted measurement within a shielded box. The power values measured within this shielded box are normalized with the values measured previously in an anechoic chamber.
The test methods include conducted and normalized setups, and utilize a range of instruments at various levels of sophistication, including signal generator, spectrum analyzer, switching modules, wireless connectivity tester, RF shielded box, equipments under test, and anechoic test chamber. Some key parameters associated with the considerations for regulatory testing are outlined, and the procedure and limitations of the different test cases are discussed. The thesis also covers the software design used for the automation of the regulatory test equipment. The purpose of performing conducted measurements is to estimate how much-normalized result deviates from the corresponding conducted result. In this study, we conclude that the measurement uncertainty for the conducted and normalized approach are on par with each other. 




















