

\newpage
\thispagestyle{empty}
%\begin{framed}
\begin{center}
\Huge\emph{Abstract}
\end{center}
\medskip
\noindent

The upcoming mobile communication standard 5G NR, which is developed by the 3GPP, uses frequencies in the millimeter-wave range.
At these short wavelengths, conventional test methods occupying cable connectors in the user equipment are impractical due to the high degree of integration of modern active antenna systems.
Therefore, in the future, radiated spurious emissions measurements and conformity tests will be carried out over the air, using the metric total radiated power.
The electromagnetic emission limits are investigated comparing the European Union and the United States of America.
First theoretical backgrounds in the required fields of antenna theory, spatial sampling, spherical sampling grids, spherical quadratures, stochastic and regression are declared.\\

Within the scope of this thesis all aspects of over the air radiation measurements are investigated.
Starting with the choice of test side, to have sufficient quiet zone size and dynamic range, calibrating a test side, evaluating the dynamic range using the link budget and choosing the ideal detector setting for the used instrument.
Subsequently, a concrete proposal for the ideal implementation of the total radiated power measurement is developed.
Including finding the minimum required measurement distance and probe antenna to provide an easy over the air total radiated power measurement.
This is accomplished by setting up a simulation of antenna under test, probe antenna and measurement radius.
Using different sets of parameters a dedicated tendency, that low directivity probes lead to less error at the same distance, is found and proofed by a measurement campaign.\\

Additionally the spherical grid type and measurement density is researched using a statistical simulation occupying rectangular arrays with random steering vectors.
It is important to have easy feasible grid types combined with as less as possible samples on the sphere to accomplish a given precision in the shortest possible time period.
Here the developed Correction Factor Reference Angle is proofed to be the optimal method to derive a aperture independent measurement error using the ideal constant step size grid with sparse sampling at the poles.
This has been shown by a regression.

