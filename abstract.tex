

\newpage
\thispagestyle{empty}
%\begin{framed}
\begin{center}
\Huge\emph{Abstract}
\end{center}
\medskip
\noindent

To Do
%Until now, nearly all coexistence measurements in 2.4 and 5 GHz ISM bands are done connecting the device under test to the test system using cables. In future, many DUTs will
%not have cable connectors any more. One technique to do these measurements is Normalized Measurements. In this study, we annalyze the measurment uncertainity for 
%
%
%Investigating adequate test methodologies in order to allow innovative use of wireless technologies while addressing wireless coexistence concerns is the goal of this novel wireless normalized measurement approach. Interference to intended communications grows with the number of wireless users, which poses a risk to the proper operation of safety-critical devices for example in the health-care industry. Nowadays, Internet of Things (IoT) devices are the fastest growing group in wireless devices. Thus, ensuring performance in the real world means testing for coexistence is a must.
%
%Wireless coexistence testing helps to mitigate the risk of interference. ETSI and FCC standards for example have defined several methodologies for testing the ability of a device to operate in a congested radio environment. Since new communication devices don’t have external connectors to their antennas we try to achieve wireless measurements that do not require connecting the Device Under Test (DUT) to the instruments. Our goal is to implement some standards methodologies, in a normalized way, so that we achieve accurate results and minimize measurement uncertainties between the conducted and the normalized approach and discuss the most important considerations for a meaningful coexistence test.
%
%%% lines to use
%The purpose of the... to estimate how much a measureed part deviates from the corresponding perfect ..
%
%This tthesis serves as a repository and comprehensive overview of the problems with... 
%
%The paper also covers new work ..
%
%Efficient organization of measurement tasks requires the knowledge and characterization of the
%parameters and of the effects that may affect the measurement itself. Uncertainty analysis is
%an example of how measurement accuracy is often difficult to quantify.
%
%The thises is focussed on this aspect of ..
%
%Wireless coexistence is a growing concern, given the ubiquity of wireless technology. Although IEEE Standards have started to address this problem in an analytical framework, a standard experimental setup and process to evaluate wireless coexistence is lacking
%
