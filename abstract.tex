

\newpage
\thispagestyle{empty}
%\begin{framed}
\begin{center}
\Huge\emph{Abstract}
\end{center}
\medskip
\noindent As more and more manufacturers incorporate wireless communication into devices, the interference to purposeful communication grows, which gives rise to a risk to the proper operation of critical devices such as wireless medical devices. Wireless coexistence, is concerned with a wireless device's ability to perform a task in a given shared space where other wireless systems are performing their tasks and may or may not be using the same set of regulations. Wireless coexistence testing therefore helps to mitigate the risk of interference and provide safe products for users worldwide. This thesis discusses the novel approach for coexistence testing for devices with and without cable connectors and is proposed to perform for 2.4 GHz ETSI standard. The technologies that the system should be evaluated against include Wi-Fi\texttrademark{}, Bluetooth\textregistered{} and others. This work is proposed only for In-Band testing and it determines the signal-to-noise threshold where wireless communications are degraded as well as the threshold where communications stop. The test methods include conducted and normalized setups, and utilize a range of instruments at various levels of sophistication, including signal generator, spectrum analyzer, switching modules, wireless connectivity tester, RF shielded box, equipments under test, and anechoic test chamber. Some key parameters associated with the considerations for regulatory testing are outlined, and the procedure and limitations of the different test cases are discussed. The thesis also covers the software design used for the automation of the regulatory test equipment. Finally, some ideas on research needed to reconcile the outputs from the various tests are explored. The purpose of performing conducted measurements is to estimate how much a measured normalized result deviates from the corresponding conducted perfect result. In this study, we conclude that the measurement uncertainty for conducted and normalised approach are on par with each other. 










