\chapter{Conclusion}

The achieved findings of this work shall be described in this chapter. Furthermore an outlook for future work will be drawn.

\section{OTA Basics}

Every part of an \acf{OTA} \acf{TRP} measurement contributes to the resulting measurement error. The very first step is to establish a tolerable error. With that the \acf{QZ} size of the \ac{OTA} chamber can be determined. This depends of the probe and the chamber it self.\\
The next error source is thermal noise, which leads to a needed dynamic range and thereby to a \acf{LB}. The measurement error induced by noise can be computed as 

\begin{equation}
e = 10\log_{10}\left(1+10^{\frac{\text{SNR}}{10}}\right).
\end{equation}

For example a \acf{SNR} of $>\SI{10}{\decibel}$ leads to a measurement error $<\SI{0.5}{\decibel}$. The received power is dependent from the radiated \ac{EIRP}, the \acf{FSPL} and the probes realized gain. To increase dynamic range a \acf{LNA} or a mixer at the probe can be used. The \ac{SNR} is further reduced by the noise figure of the used equipment. A typical modern spectrum analyzer has a noise figure of $\approx\SI{15}{\decibel}$ for example. to increase the \ac{SNR} the \ac{FSPL} should be minimized by occupying the shortest sensible measurement range.

\section{Measurement Distance}

In the framework of this thesis minimum measurement distance for \acf{TRP} measurements is examined. The simulation shows that \ac{TRP} measurements can be done using low directivity probe antennas in the radiating \acf{NF} at $\approx$ Derat distance. Furthermore it is shown that the \ac{TRP} measurement error is positively correlated with the directivity of the probe antenna. Taking low directivity probe antennas leads to smaller \ac{OTA}-measurement-chambers and results in higher dynamic range and lower costs. This could also be shown in a real measurement campaign.

\section{Spatial Sampling}

Spatial sampling error is produced by aliasing. In this thesis this subject is addressed and it is shown that the ideal spherical sampling grid is the \acf{CSSG} because it's flexibility and it's efficient realization with spars sampling a the poles and swept azimuth. It is further demonstrated that that the introduced \ac{CrefA} factor is the ideal compromise between the two sampling approaches from \cite{hansen} and \cite{2018arXiv180310993F} with a \acf{SF} of one.

\section{Outlook}

To generate more data to support the statement that low directivity probes enable closer sampling at the same error, more real measurements should be done. E.q. the mounting of the patch antenna could be optimized, so that it is slightly tilted and the reflected \acf{EM}-wave is scattered towards the absorber. Additionally \ac{TRP} measurements with a \ac{OEWG} instead of a WR28 waveguide adapter should be done as well.\\
Moreover the spatial sampling simulation should be done with the measurement data from several real antenna arrays with random steering vectors to get an empirical proof of the findings. In addition real measurements must be done, this will indeed be exhaustive caused by the high number of samples to generate statistical relevance.