\chapter{Conclusion} \label{chap:8}

The achieved findings of this work shall be described in this chapter. Furthermore, an outlook for future work will be drawn.

\section{Achievements}
This work presents a coexistence testing solution for devices with or without antenna ports, based on a novel regulatory test system that has a compact size and high quality shielding and allows lower coupling losses. The project implements normalized and conducted measurements to perform the device-testing, and investigates the \acf{MU} for both cases. To this end, an algorithm for the optimization of antenna pairs is implemented on both the simulation and the measurements. After that, the hardware and software implementations for \acf{OBW}, \acf{PSD} and receiver blocking test cases are done based on the standard ETSI EN 300 328 V2.1.1. \\

 The results for both single and multi-antennas \acsp{DUT} showed that, comparable \acf{MU} results to the conducted measurements can be achieved by the normalized approach. While studying the \acf{MU} results for the presented test cases, many factors may influence the accuracy of the test results. In addition to the errors caused by thermal noise, there are interferences and cross-talk between the measurement antennas that can influence the results. Also, the results of the \acf{MU} for single-antennas \acs{DUT} are proved to be better than those of multi-antennas. \\
 
 Finally, in-band testing determines the performance of wireless devices and the thresholds where wireless communications are degraded as well as the threshold where communications stop i.e where the limits set by the standards are exceeded. The report of these tests provides the necessary data to make a more informed risk analysis of the radio's behavior and to enable troubleshooting the reported interference problems. The results prove that performing the normalized measurements doesn't result in a major degrading the quality of the evaluation.


\section{Outlook}
Further investigations and research are possible to improve this work; for the optimization of the antenna pairs connections within the test chamber, the simulation can be improved by including other effects that can increase the accuracy of the simulation to match the choice based on actual measurements such as minor reflections in the chamber and isolation between the probe antennas. The simulation can also be done with multi-antenna \acsp{DUT}. \\

 In addition, the test-cases must be implemented for 5 GHz and the long-measurements must be done with repositioning the \acs{DUT} within the test chamber. The same work can also be implemented for the standard ETSI EN 301893 \cite{etsi301893}. Increasing the solution for more test cases must also be planned.  Account for isolation (cross-talk) between probe antennas during selection of best probe antenna can be another important parameter for research. \\
 
 \acs{EIRP} measurement within the \acf{AC} is also very important and must be performed. Conducted measurements for multi-antenna \acs{DUT} must be executed to compare the results with the outcome from normalized measurements. 


