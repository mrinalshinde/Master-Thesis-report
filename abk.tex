%%%%%%%%%%%%%%%%%%%%%%%%%%%%%%%%%%%%%%%%%%%%%%%%%%%%%%%%%%%%%%%%%%%%%%%%%%%%%%
%% Descr:       Vorlage für Berichte der DHBW-Karlsruhe, Datei mit Abkürzungen
%% Author:      Prof. Dr. Jürgen Vollmer, vollmer@dhbw-karlsruhe.de
%% $Id: abk.tex,v 1.3 2016/03/16 12:21:40 vollmer draft $
%% -*- coding: utf-8 -*-
%%%%%%%%%%%%%%%%%%%%%%%%%%%%%%%%%%%%%%%%%%%%%%%%%%%%%%%%%%%%%%%%%%%%%%%%%%%%%%%

\chapter*{List of Abbreviations}                   % chapter*{..} -->   keine Nummer, kein "Kapitel"
						         % Nicht ins Inhaltsverzeichnis
% \addcontentsline{toc}{chapter}{Akürzungsverzeichnis}   % Damit das doch ins Inhaltsverzeichnis kommt

% Hier werden die Abkürzungen definiert
\begin{acronym}[FIAFTA]
  % \acro{Name}{Darstellung der Abkürzung}{Langform der Abkürzung}
 \acro{Abk}[Abk.]{Abkürzung}

 % Folgendes benutzen, wenn der Plural einer Abk. benöigt wird
 % \newacroplural{Name}{Darstellung der Abkürzung}{Langform der Abkürzung}
 \newacroplural{Abk}[Abk-en]{Abkürzungen}

 \acro{H2O}[\ensuremath{H_2O}]{Di-Hydrog-Monoxid}

 % Wenn neicht benutzt, erscheint diese Abk. nicht in der Liste
 \acro{NUA}{Not Used Acronym}
 
 \acro{3GPP}{3rd Generation Partnership Project}
 \acro{5G}{Fifth Generation mobile communication}
 \acro{AAS}{Active Antenna System}
 \acro{AC}{Anechoic Chamber}
 \acro{ANSI}{American National Standards Institute}
 \acro{ATS}{Antenna Test Site}
 \acro{CATR}{Compact Antenna Test Range}
 \acro{CDG}{Constant Density Grid}
 \acro{CEPT}{European Conference of Postal and Telecommunications Administrations}
 \acro{CrefA}{Correction Factor Reference Angle}
 \acro{CSSG}{Constant Step Size Grid}
 \acro{DANL}{Displayed Average Noise Level}
 \acro{DUT}{Device Under Test} 
 \acro{EBW}{Emission Bandwidth}
 \acro{EIRP}{Effective Isotropic Radiated Power}
 \acro{EM}{Electro Magnetic}
 \acro{EMC}{Electro Magnetic Compatibility}
 \acro{ETSI}{European Telecommunications Standards Institute}
 \acro{EU}{European Union}
 \acro{FCC}{Federal Communications Commission}
 \acro{FF}{Far-Field}
 \acro{FFT}{Fast Fourier Transformation}
 \acro{FR2}{Frequency Range two}
 \acro{FSPL}{Free Space Path Loss}
 \acro{IEEE}{Institute of Electrical and Electronics Engineers}
 \acro{ITU}{International Telecommunication Union}
 \acro{LB}{Link Budget}
 \acro{MIMO}{Multiple In Multiple Out}
 \acro{mmW}{milli-meter Wave}
 \acro{NF}{Near-Field}
 \acro{NR}{New Radio}
 \acro{OATS}{Open Area Test Site}
 \acro{OBW}{Occupied Bandwidth}
 \acro{OOB}{Out Of Band Emission}
 \acro{OTA}{Over The Air}
 \acro{PM}{Pattern Multiplication}
 \acro{QZ}{Quite Zone}
 \acro{RBW}{Resolution Bandwidth}
 \acro{RF}{Radio Frequency}
 \acro{RMS}{Root Mean Square}
 \acro{RS}[R\&{}S]{Rohde \& Schwarz GmbH \& Co. KG}
 \acro{RSE}{Radiated Spurious Emission} 
 \acro{RVC}{Reverberation Chamber}
 \acro{SE}{Spurious Emission}
 \acro{SF}{Sparsity Factor}
 \acro{SGH}{Standard Gain Horn}
 \acro{TRP}{Total Radiated Power}
 \acro{TSP}{Travelling Salesman Problem}
 \acro{UE}{User Equipment}
 \acro{UnE}{Unwanted Emission}
 \acro{USA}{United States of America}
 \acro{VNA}{Vector Network Analyzers}
 \acro{VSG}{Vector Signal Generator}


\end{acronym}