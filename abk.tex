%%%%%%%%%%%%%%%%%%%%%%%%%%%%%%%%%%%%%%%%%%%%%%%%%%%%%%%%%%%%%%%%%%%%%%%%%%%%%%
%% Descr:       Vorlage für Berichte der DHBW-Karlsruhe, Datei mit Abkürzungen
%% Author:      Prof. Dr. Jürgen Vollmer, vollmer@dhbw-karlsruhe.de
%% $Id: abk.tex,v 1.3 2016/03/16 12:21:40 vollmer draft $
%% -*- coding: utf-8 -*-
%%%%%%%%%%%%%%%%%%%%%%%%%%%%%%%%%%%%%%%%%%%%%%%%%%%%%%%%%%%%%%%%%%%%%%%%%%%%%%%

\chapter*{List of Abbreviations}                   % chapter*{..} -->   keine Nummer, kein "Kapitel"
						         % Nicht ins Inhaltsverzeichnis
% \addcontentsline{toc}{chapter}{Akürzungsverzeichnis}   % Damit das doch ins Inhaltsverzeichnis kommt

% Hier werden die Abkürzungen definiert
\begin{acronym}[MOSFET]
  % \acro{Name}{Darstellung der Abkürzung}{Langform der Abkürzung}
 \acro{Abk}[Abk.]{Abkürzung}

 % Folgendes benutzen, wenn der Plural einer Abk. benöigt wird
 % \newacroplural{Name}{Darstellung der Abkürzung}{Langform der Abkürzung}
 \newacroplural{Abk}[Abk-en]{Abkürzungen}

 \acro{H2O}[\ensuremath{H_2O}]{Di-Hydrog-Monoxid}

 % Wenn neicht benutzt, erscheint diese Abk. nicht in der Liste
 \acro{NUA}{Not Used Acronym}
 
\acro{DUT}{Device Under Test} 
 
 
 \acro{AC}{Anechoic Chamber}
 \acro{EMC}{Electro Magnetic Compatibility}
 \acro{EM}{Electro Magnetic}
 \acro{EIRP}{Effective Isotropic Radiated Power}
 \acro{CATR}{Compact Antenna Test Range}
 \acro{CSSG}{Constant Step Size Grid}
 \acro{CDG}{Constant Density Grid}
 \acro{FFT}{Fast Fourier Transformation}
 \acro{RSE}{Radiated Spurious Emission} 
 \acro{RVC}{Reverberation Chamber}
 \acro{SE}{Spurious Emission}
 \acro{UnE}{Unwanted Emission}
 \acro{OOB}{Out Of Band Emission}
 \acro{OATS}{Open Area Test Site}
 \acro{EBW}{Emission Bandwidth}
 \acro{RMS}{Root Mean Square}
 \acro{TRP}{Total Radiated Power}
 \acro{TSP}{Travelling Salesman Problem}
 
 \acro{OTA}{Over The Air}
 
 \acro{FF}{Far-Field}
 \acro{NF}{Near-Field}
 
 \acro{SGH}{Standard Gain Horn}



\end{acronym}